\documentclass{beamer}
\usepackage{caption}
\usepackage{subcaption}
\usepackage{../arbenson-math}
\usepackage{dgleich-math}
\usepackage{wasysym}

\usetheme{boxes}
\usecolortheme{seahorse}

\setbeamertemplate{frametitle}
{
    \nointerlineskip
    \begin{beamercolorbox}[sep=0.3cm,ht=1.8em,wd=\paperwidth]{frametitle}
        \vbox{}\vskip-2ex%
        \strut\insertframetitle\strut \hfill \insertframenumber
        \vskip-0.8ex%
    \end{beamercolorbox}
}

\AtBeginSection[]
{
  \begin{frame}<beamer>
    \frametitle{\insertsection}
    \tableofcontents[currentsection]
  \end{frame}
}

\title{Tall-and-skinny Matrix Computations \\
 in MapReduce}
\author{
Austin Benson \\
Institute for Computational and Mathematical Engineering \\
Stanford University \\
\vspace{0.2in}
SIAM PP 2014 \\
}
\date{February 19, 2014}
\begin{document}
\maketitle

\begin{frame}
\frametitle{Tall-and-skinny matrices}

What are tall-and-skinny matrices?

\begin{figure}[h!]
\centering
\includegraphics[height=1.5in]{"figs/ts_mat"}
\end{figure}

$A$ is $m \times n$ and $m >> n$.  \\
Examples: rows are data samples; blocks of $A$ are images from a video; Krylov subspaces; Unrolled tensors

\end{frame}


\begin{frame}
\frametitle{Matrix representation}
Maybe the data is a set of samples

\vspace{0.2in}

\[
A = \bmat{1.0 & 0.0 \\ 2.4 & 3.7 \\ 0.8 & 4.2 \\ 9.0 & 9.0} \rightarrow \bmat{ (\text{``Apr 26 04:18:49"}, [1.0, 0.0]) \\  (\text{``Apr 26 04:18:52"}, [2.4, 3.7]) \\ (\text{``Apr 26 04:19:12"}, [0.8, 4.2]) \\ (\text{``Apr 26 04:22:33"}, [9.0, 9.0]) }
\]

\vspace{0.3in}
(key, value) $\rightarrow$ (timestamp, sample)

\end{frame}

\begin{frame}
\frametitle{Matrix representation: an example}
Scientific example: (x, y, z) coordinates and model number: \\

\vspace{0.1in}

\texttt{(\footnotesize{(47570,103.429767811242,0,-16.525510963787,iDV7)}, 
\scriptsize{[0.00019924 -4.706066e-05 2.875293979e-05 2.456653e-05 -8.436627e-06 -1.508808e-05 3.731976e-06 -1.048795e-05 5.229153e-06 6.323812e-06])}}

\begin{figure}
        \centering
	\begin{subfigure}[b]{0.48\textwidth}
	\centering
	\includegraphics[height=1.9in]{"figs/pressure"}
	\end{subfigure}
        ~ %add desired spacing between images, e. g. ~, \quad, \qquad etc. 
          %(or a blank line to force the subfigure onto a new line)
	\begin{subfigure}[b]{0.48\textwidth}
	\centering
	\includegraphics[height=1.9in]{"figs/leverage3"}
	\end{subfigure}
	\caption{Aircraft simulation data.}
\end{figure}
\end{frame}

\begin{frame}
\frametitle{Tall-and-skinny QR}
\begin{figure}
\centering
\includegraphics[width=2in]{"figs/ts_qr"}
\end{figure}

Tall-and-skinny (TS): $m >> n$.  $Q^TQ = I$.
\end{frame}

\begin{frame}
\frametitle{TS-QR $\rightarrow$ TS-SVD}

\begin{figure}[h!]
\centering
\includegraphics[height=2in]{"figs/ts_qr_svd"}
\end{figure}

\vspace{0.2in}

$R$ is small, so computing its SVD is cheap.
\end{frame}


\begin{frame}
\frametitle{Why Tall-and-skinny QR and SVD?}
\begin{enumerate}
\setlength{\itemsep}{0.05in}
\item{Regression with many samples}
\item{Principle Component Analysis (PCA)}
\item{Model Reduction}
\end{enumerate}

\begin{figure}[h!]
\centering
\includegraphics[width=3in]{"figs/jet_labelled"}
\caption{Dynamic mode decomposition of the screech of a jet.
Joe Nichols, University of Minnesota.}
\end{figure}
\end{frame}

\begin{frame}
\frametitle{Communication-avoiding TSQR}

\beqstar
\mA = \underbrace{\bmat{\mA_1 \\ \mA_2 \\ \mA_3 \\ \mA_4} }_{8n \times 4n}
&=& \underbrace{\bmat{ \mQ_1 \\ & \mQ_2 \\ & & \mQ_3 \\ & & & \mQ_4 }}_{8n \times 4n}
      \underbrace{\bmat{ \mR_1 \\ \mR_2 \\ \mR_3 \\ \mR_4}}_{4n \times n} \\
&=& \overbrace{\underbrace{\bmat{ \mQ_1 \\ & \mQ_2 \\ & & \mQ_3 \\ & & & \mQ_4 }}_{8n \times 4n}
       \underbrace{\tilde{\mQ}}_{4n \times n}}^{= \mQ} 
       \quad
       \vphantom{\bmat{ \mQ_1 \\ & \mQ_2 \\ & & \mQ_3 \\ & & & \mQ_4 }}\underbrace{\mR}_{n \times n}
\eeqstar

[Demmel, Grigori,  Hoemmen, Langou 2008]
\end{frame}


\begin{frame}
\frametitle{Communication-avoiding TSQR}
\[
\mA = \underbrace{\bmat{\mA_1 \\ \mA_2 \\ \mA_3 \\ \mA_4} }_{8n \times 4n}
= \underbrace{\bmat{ \mQ_1 \\ & \mQ_2 \\ & & \mQ_3 \\ & & & \mQ_4 }}_{8n \times 4n}
      \underbrace{\bmat{ \mR_1 \\ \mR_2 \\ \mR_3 \\ \mR_4}}_{4n \times n}
\]

\vspace{0.4in}

\begin{itemize}
\item $\mA_i = Q_iR_i$ can be computed in parallel.
\item Only need $R$ (or $\Sigma$) $\to$ throw out $Q_i$
\end{itemize}
\end{frame}

\begin{frame}
\frametitle{MapReduce TSQR}

\begin{figure}[h!]
\centering
\includegraphics[height=2in]{"figs/mr_tsqr"}
\caption{$S^{(1)}$ is the matrix consisting of the rows of all of the $R_i$ factors.  Similarly, $S^{(2)}$ consists of all of the rows of the $R_{2, j}$ factors.}
\end{figure}

[Constantine and Gleich 2011]

\end{frame}


\begin{frame}
\frametitle{MapReduce TSQR: the implementation}
\lstset{basicstyle=\tiny}
\lstset{numbers=left}
\codeblock{code/mrtsqr.py}

\end{frame}


\begin{frame}
\frametitle{MapReduce TSQR: Getting $Q$}
We have an efficient way of getting $R$ in $QR$ (and $\Sigma$ in the SVD).  What if I want $Q$ (or my left singular vectors)?

\vspace{0.2in}

\[
A = QR \rightarrow Q = AR^{-1}
\]

\vspace{0.2in}

\begin{itemize}
\item \textcolor{problem}{Problem}:  $Q$ can be far from orthogonal.
\item \textcolor{theblue}{Numerically stable solution}: more complicated code, $\sim$15\% performance penalty
[Benson, Gleich, Demmel 2013]
\end{itemize}

\end{frame}


\begin{frame}
\frametitle{Stability}

\begin{figure}[h!]
\centering
\includegraphics[height=2.5in]{"figs/stability"}
\end{figure}

\end{frame}


\begin{frame}
\frametitle{Questions}

Questions?

\vspace{0.2in}

\begin{itemize}
\setlength{\itemsep}{0.1in}
\item Austin Benson: \texttt{arbenson@stanford.edu}
\item Code: \url{https://github.com/arbenson/mrtsqr}
\end{itemize}

\end{frame}


\end{document}
